% Initial version by Darian Muresan, Ph.D.
% Edit and adjust as needed.

\documentclass[12pt]{cornell}

% add index support
\makeindex

% graphing programs
\usepackage{color}
\usepackage{psfrag}
\usepackage{verbatim}
\usepackage{fancyhdr}
%\usepackage{titlesec}
\usepackage{fancyvrb} 
% hyperlink programs
\usepackage[pdfmark, 
breaklinks=true, 
colorlinks=true,
citecolor=blue,
linkcolor=blue,
menucolor=black,
pagecolor=black,
urlcolor=blue
]{hyperref} % links in pdf
%\usepackage[colorlinks]{hyperref} % links in dvi
\usepackage{listings}
\usepackage{amsfonts} 
\usepackage{amssymb} 
%\usepackage{tabto}

\usepackage{tabularx,colortbl}
\usepackage[chapter]{algorithm} 
\usepackage{algorithmic} 
\usepackage{blindtext}
\usepackage{imakeidx}


\definecolor{DarkGreen}{rgb}{0,0.6,0}
\definecolor{mygreen}{rgb}{0,0.6,0}
\definecolor{mygray}{rgb}{0.5,0.5,0.5}
\definecolor{mymauve}{rgb}{0.58,0,0.82}

\usepackage{tocloft}
\usepackage{amsmath}
\usepackage{tcolorbox}
\usepackage{enumitem}
\usepackage{longtable}
%\usepackage{textcomp}
\usepackage{txfonts}

%part for \part titles
%chap for \chapter titles
%sec for \section titles
%subsec for \subsection titles
%subsubsec for \subsubsection titles
%para for \paragraph titles
%subpara for \subparagraph titles
%fig for figure \caption titles
%subfig for subfigure \caption titles
%tab for table \caption titles
%subtab for subtable \caption titles

% update chapter number spacing
\setlength{\cftchapnumwidth}{2em}
\setlength{\cftsecnumwidth}{2.5em}
\setlength{\cftsubsecnumwidth}{3.5em}
\setlength{\cftsubsubsecnumwidth}{4.5em}

\addtolength{\cftsecindent}{0.5em}
\addtolength{\cftsubsecindent}{0.5em}
\addtolength{\cftsubsubsecindent}{0.5em}

%\titlespacing*{\chapter}{0pt}{-50pt}{20pt}
%\titleformat{\chapter}[display]{\normalfont\huge\bfseries}{\chaptertitlename\ 
%\thechapter}{20pt}{\Huge}
%\pagestyle{fancy}
%\pagestyle{cornell}
%
%\rhead{F054-021-0172}
%\chead{Nonlinear Enhancement of Visual Target Detection (AF05-T021)}
%\lhead{GSTI}
%\lfoot{\scriptsize Use or disclosure of data on this page is subject
%to the restriction on the title page of this proposal.}
%\cfoot{}
%\rfoot{\thepage}

\newfont{\Bp}{msbm10}
\newfont{\BpBig}{msbm10 scaled\magstep2}
\newfont{\Sc}{eusm10}
\newfont{\ScBig}{eusm10 scaled\magstep3}
\newfont{\Fr}{eufm10}
\newfont{\FrBig}{eufm10 scaled\magstep1}

% some commands:
\newcommand{\dxi}{{\tt m\_xDeltaInput}}
\newcommand{\dyi}{{\tt m\_yDeltaInput}}
\newcommand{\dci}{{\tt m\_cDeltaInput}}
\newcommand{\dxo}{{\tt m\_xDeltaOutput}}
\newcommand{\dyo}{{\tt m\_yDeltaOutput}}
\newcommand{\dco}{{\tt m\_cDeltaOutput}}
\newcommand{\ttf}[1]{{\tt #1}}
\newcommand{\tbl}[2]{{\begin{tabular}{c} #1 \\ #2 \end{tabular}}}

\newcommand{\urltwo}[2]{\mbox{\href{#1}{\tt #2}}}
\newcommand{\qnorm}[1]{\|#1\|_{\bQ}}
\newcommand{\qdot}[2]{\lrb #1, #2 \rrb_{\bQ}}
\newcommand{\kdot}[2]{\lrb #1, #2 \rrb_{\bf k}}
\newcommand{\tdot}[2]{\lrb #1, #2 \rrb}
\newcommand{\mydiff}[2]{\lrb #1 - #2 \rrb}
\newcommand{\lena}{\textit{lena}}
\newcommand{\barb}{\textit{barbara}}
\newcommand{\boat}{\textit{boat}}
\newcommand{\leaves}{\textit{leaves}}
\newcommand{\rings}{\textit{rings}}
\newcommand{\treg}{\textit{train region}}
\newcommand{\dreg}{\textit{denoise region}}
\newcommand{\oreg}{\textit{overlap region}}
\newcommand{\sil}{\sigma_l^2}
\newcommand{\sn}{\sigma^2}
\newcommand{\bn}{{\mbox{\bf \FrBig N}}}
\newcommand{\n}{\mbox{\Fr N}}
%\newcommand{\bn}{\bf N}
%\newcommand{\n}{N}
\newcommand{\bY}{\textbf{Y}}
\newcommand{\bX}{\textbf{X}}
\newcommand{\bb}{\textbf{b}}
\newcommand{\bu}{\textbf{u}}
\newcommand{\bv}{\textbf{v}}
\newcommand{\by}{\textbf{y}}
\newcommand{\bx}{\textbf{x}}
\newcommand{\be}{\textbf{e}}
\newcommand{\bz}{\textbf{z}}
\newcommand{\bs}{\textbf{s}}
\newcommand{\bw}{\textbf{w}}
\newcommand{\bQ}{\textbf{Q}}
\newcommand{\bphi}{\textbf{$\phi$}}
\newcommand{\lsb}{\left[}
\newcommand{\rsb}{\right]}
\newcommand{\lrb}{\left(}
\newcommand{\rrb}{\right)}
\newcommand{\lcb}{\left\{}
\newcommand{\rcb}{\right\}}
\newcommand{\R}{\mbox{\BpBig R}}
\newcommand{\F}{{\cal F}}
\newcommand{\Fk}{\mbox{\Sc F}}
\newcommand{\bQF}{\textbf{Q}_{\mbox{\Sc F}}}
\newcommand{\N}{{\cal N}}
\newcommand{\xlz}{X_l(z)}
\newcommand{\xhz}{X_h(z)}
\newcommand{\xz}{X(z)}
\newcommand{\pr}{ perfect reconstruction }
\newcommand{\smb}{Smith-Barnwell }
\newcommand{\xw}{X(e^{j\omega})}
\newcommand{\xmw}{X(-e^{j\omega})}
\newcommand{\dw}{D(e^{j\omega})}
\newcommand{\dmw}{D(-e^{j\omega})}
\newcommand{\ew}{E(e^{j\omega})}
\newcommand{\emw}{E(-e^{j\omega})}
\newcommand{\fw}{F_0(e^{j\omega})}
\newcommand{\fmw}{F_0(-e^{j\omega})}
\newcommand{\hoz}{H_1(z)}
\newcommand{\hzz}{H_0(z)}
\newcommand{\goz}{G_1(z)}
\newcommand{\gzz}{G_0(z)}
\newcommand{\hzw}{H_{0}(e^{j\omega})}
\newcommand{\hzmw}{H_{0}(-e^{j\omega})}
\newcommand{\hzcw}{H_{0}(e^{-j\omega})}
\newcommand{\how}{H_1(e^{j\omega})}
\newcommand{\homw}{H_1(-e^{j\omega})}
\newcommand{\gzw}{G_0(e^{j\omega})}
\newcommand{\gzmw}{G_0(-e^{j\omega})}
\newcommand{\gow}{G_1(e^{j\omega})}
\newcommand{\gomw}{G_1(-e^{j\omega})}
\newcommand{\wl}{e^{-jwL}}
\newcommand{\aqua}{\textit{AQua with OR }}
\newtheorem{theorem}{Theorem}
\newtheorem{lemma}{Lemma}
\newtheorem{corollary}{Corollary}
\newtheorem{claim}{Claim}
\newtheorem{definition}{Definition}
\newenvironment{proof}{\noindent{\em Proof.}}{\ \hfill Q.E.D.}
%\newtheorem{moduleCount}{L}
\newcommand*{\labelfile}[1]{%
  \label{file:#1}%
}

\lstset{ %
  backgroundcolor=\color{white},   % choose the background color; you must add \usepackage{color} or \usepackage{xcolor}
  basicstyle=\footnotesize,        % the size of the fonts that are used for the code
  breakatwhitespace=false,         % sets if automatic breaks should only happen at whitespace
  breaklines=true,                 % sets automatic line breaking
  captionpos=b,                    % sets the caption-position to bottom
  commentstyle=\color{DarkGreen},    % comment style
  deletekeywords={...},            % if you want to delete keywords from the given language
  escapeinside={\%*}{*)},          % if you want to add LaTeX within your code
  extendedchars=true,              % lets you use non-ASCII characters; for 8-bits encodings only, does not work with UTF-8
  %frame=single,                   % adds a frame around the code
  keepspaces=true,                 % keeps spaces in text, useful for keeping indentation of code (possibly needs columns=flexible)
  keywordstyle=\color{blue},       % keyword style
  language=C++,                    % the language of the code
  morekeywords={*,...},            % if you want to add more keywords to the set
  numbers=left,                    % where to put the line-numbers; possible values are (none, left, right)
  numbersep=5pt,                   % how far the line-numbers are from the code
  numberstyle=\tiny\color{mygray}, % the style that is used for the line-numbers
  rulecolor=\color{black},         % if not set, the frame-color may be changed on line-breaks within not-black text (e.g. comments (green here))
  showspaces=false,                % show spaces everywhere adding particular underscores; it overrides 'showstringspaces'
  showstringspaces=false,          % underline spaces within strings only
  showtabs=false,                  % show tabs within strings adding particular underscores
  stepnumber=1,                    % the step between two line-numbers. If it's 1, each line will be numbered
  stringstyle=\color{mymauve}     % string literal style
  %tabsize=2,                      % sets default tabsize to 2 spaces
  %caption=\lstname                % show the filename of files included with \lstinputlisting; also try caption instead of title
}

% Uncomment draftcopy to get the word DRAFT boldly across the first page
%   By the way, xdvi won't show it but it will come out when you print
%\usepackage[light,all]{draftcopy}		% DRAFT on first page
%\draftcopySetGrey{.97}
%\draftcopyName{Confidential}{150}
%\draftcopFirstPage{1}

% Uncomment drafthead to get the date and DRAFT in the header of pages
% that are normallly numbered on the top, pages 2-n of each chapter for example
% This doesn't work with centered page numbers: \pagestyle{cornellc}
%\usepackage{drafthead}

% Including selective chapters:
% use this to selectively process chapters, etc.  Put a % in front of
% the sections that you don't want done this time.  Includes are
% used instead of \input so that LaTeX will keep track of chapters and
% pages without processing everything.  Don't let any spaces creep in
% around the words or it will not work!


\includeonly{
prologue,
manIntroduction
}

\makeindex

\begin{document}

\pagenumbering{roman}
\singlespacing
% File: prologue.tex
% Thesis prologue:  Title page, acknowledgements, table of contents,
% list of figures, and list of tables.
%
% this file is to be \include'd after the \begin{document}

% Cornell-style title page
\begin{titlepage}
        \title{In-Hospital Management System}
        \author{Team \\ Stevens.edu }
        \conferraldate{}{\today} \maketitle
\end{titlepage}

% Copyright page
%\begin{copyrightpage}
\makecopyright
%\end{copyrightpage}

% Abstract: the abstract body is pulled from the file abstract.tex;
%  the title is pulled from the \title command in the titlepage section
\begin{abstract}
        %\makeabstitle
        \input abstract      % puts the abstract file here
\end{abstract}

% Biographical information pulled from file bio.tex
%\begin{biosketch} \input bio \end{biosketch}

% Dedication (optional):  pulls information from file dedication.tex
%\begin{dedication} 
%\input dedicate 
%\end{dedication}

% Acknowledgements:  pulls information from file acknow
%\begin{acknowledgements} \input acknow \end{acknowledgements}

% Table of contents
\contentspage

% If you have no tables or figures put a % in front of the list page line
% List of tables
\tablelistpage

% List of figures
\figurelistpage



\setcounter{page}{1}        % set page counter
\pagenumbering{arabic}      % set page number style
\pagestyle{fancy}         % top right page numbers
%\pagestyle{cornell}
%\pagestyle{cornellc}       % centered page numbers, disables drafthead

\renewcommand{\chaptermark}[1]{\markboth{#1}{}}
\renewcommand{\sectionmark}[1]{\markright{#1}{}}

\fancyhead{} % clear all fields

\lhead{Chapter \thechapter}
%\lhead{\thechapter}
\chead{\leftmark}
\rhead{\thepage}


\lfoot{Chapter \thechapter}
\cfoot{\copyright Stevens -- \today \mbox{} -- Project Name}
\rfoot{\thepage}

\renewcommand{\headrulewidth}{0.4pt}
\renewcommand{\footrulewidth}{0.4pt}

%\rhead{F054-021-0172}
%\chead{Nonlinear Enhancement of Visual Target Detection (AF05-T021)}
%\lhead{GSTI}
%\lfoot{\scriptsize Use or disclosure of data on this page is subject
%to the restriction on the title page of this proposal.}
%\cfoot{}
%\rfoot{\thepage}


\singlespacing
\chapter{Introduction \\
\small{\textit{-- Team}} 
\index{Chapter!introduction}
\index{introduction}
\label{Chapter::Introduction}}

% Add a section and label it so that we can reference it later
\section{The Team \label{Section::The Team}}

We are a team of three students and below is a biref introduction about us.\\
\textbf{Tanay Shah:}

I am currently pursuing Master's in Computer Science and this is my 2nd Semester. I completed my Bachelors of Computer Engineering in 2020. I worked as a Software Developer for 2 years in an IT firm located in Gujarat, India. I also did freelancing work for a year which included 15 projects. I have good amount of experience in PHP, Java, Python, JavaScript, NodeJS, Laravel Framework, Zend Framework, AWS Management.

\textbf{Chinamy Dali:}

I am currently pursuing a Master's in Computer Science at Stevens Institute of Technology. I have an experience of just over an year in .NET, java spring and nodejs as a full-stack developer; building web-apps and maintaining large scale backends. I am interested in programming languages and compilers, which have been my focus for my graduate studies. I have been built an interpreter for my own language using OCaml and currently pursing Compiler Design course build a X86 architecture targeted compiler for the same language. 

\textbf{Moksha Dave}

I am pursuing my Master's degree with Computer Science as my major. It's my second semester at Stevens and I've been enjoying every part of it. Coming to my bachelor's, I completed my BE in Information Technology from Gujarat Technological University (GTU) during which I studied courses like Data structures, algorithms, computer networks, operating systems, AI, Computer Vision, Data Visualization and a few more. Overall, I am a fast and an enthusiastic learner and a reliable one as well! Something that people wouldn't know about me yet is, I will give in my 110\% once I decide on something.

 \section{Stakeholders for in-Hospital Management System}
 \subsection{Patients and their families:}They are the primary users of the system and have a direct interest in its functionality and reliability.
 \subsection{Hospital administrators:} They are responsible for managing the hospital's resources, including the management system, and are interested in its efficiency, effectiveness, and cost-effectiveness.
 \subsection{Clinical staff:} Doctors, nurses, and other clinical staff members use the system to manage patient information, schedules, and treatments, and are concerned with its accuracy and ease of use.
 \subsection{IT staff:} They are responsible for the installation, maintenance, and support of the system, and are interested in its technical feasibility and scalability.
 \subsection{Insurance providers:} They have an interest in the information recorded in the system, as it is used to determine patient coverage and payment for services.
 \subsection{Regulatory agencies:} They monitor the quality of healthcare services and have an interest in the accuracy and completeness of patient information recorded in the system.
 \subsection{Finance and accounting departments:} They are concerned with the financial aspects of the hospital, including budgeting, billing, and revenue management. The In-Hospital Management System plays a critical role in these processes and they have an interest in its financial reporting capabilities.

 \section{Possible conflicts between stakeholders}
\subsection{Patients and hospital administrators:} Patients may prioritize convenience and accessibility of healthcare services, while administrators may prioritize cost-effectiveness and resource utilization. For example, patients may prefer to have more testing or treatment options available, while administrators may seek to limit the number of tests or treatments to control costs.
\subsection{Clinical staff and IT staff:} Clinical staff may prioritize the ease of use and reliability of the system, while IT staff may prioritize security and scalability. For example, clinical staff may request more user-friendly interface, while IT staff may focus on implementing stronger security measures.
\subsection{Insurance providers and hospital administrators:} Insurance providers may prioritize cost control and cost sharing, while administrators may prioritize providing the best possible healthcare services to patients. For example, insurance providers may seek to limit the number of tests or treatments covered by insurance, while administrators may seek to provide more comprehensive care to patients.
\subsection{Regulatory agencies and hospital administrators:} Regulatory agencies may prioritize patient safety and quality of care, while administrators may prioritize cost-effectiveness and resource utilization. For example, regulatory agencies may require more strict reporting and documentation standards, while administrators may seek to streamline processes to reduce costs.
\subsection{Clinical staff and supply chain management:} Clinical staff may prioritize the availability of necessary medical supplies and equipment, while supply chain management may prioritize cost control and resource utilization. For example, clinical staff may request additional equipment or supplies to improve patient care, while supply chain management may seek to limit the number of items purchased to control costs.


% add a new page
\newpage
Hi there world!  Here is an example of a note\footnote{Here is a reference 
to Figure \ref{Figure::manAgile} and an indexed keyword\index{keyword}.}
\chapter{Vision and Scope Document for In-Hospital Management  \\
\small{\textit{-- Team}} 
\index{Chapter!introduction}
\index{introduction}
\label{Chapter::Introduction}}

% Add a section and label it so that we can reference it later
\section{Business Requirements\label{Section::Business Requirements}}
\subsection{Patient Management} The system should be able to manage patient registration, appointment scheduling, medical history, lab results, prescriptions, and billing.
\subsection{Staff Management: }The system should be able to manage staff scheduling, payroll, performance evaluation, and training.
\subsection{Inventory Management: } The system should be able to manage the hospital's inventory of medical supplies, equipment, and drugs.
\subsection{Medical Records Management:}  The system should be able to manage patient medical records, including diagnosis, treatment, and test results.
\subsection{Communication and Collaboration} The system should facilitate communication and collaboration between staff members, such as messaging, shared calendars, and task assignment.
\subsection{Integration and Interoperability:} The system should be able to integrate with other hospital systems, such as electronic health records, medical imaging systems, and billing systems, and be inter-operable with other healthcare providers
\subsection{Data Security and Privacy} The system should ensure the security and privacy of patient and hospital data, including compliance with applicable regulations and standards.
\subsection{Reporting and Analytics:} The system should provide real-time reporting and analytics, such as occupancy rates, patient outcomes, and financial performance.
\pagebreak

\section{Scope and Limitations \label{Section::Scope and Limitations}}
Scope refers to the boundaries of what a project or system will include, while limitations refer to the factors that could restrict or hinder the project or system's development or use. Here are some examples of scope and limitations:
\subsection{Scope} -The scope of a project to build a new hospital might include designing and constructing the building, installing medical equipment and systems, hiring and training staff, and establishing policies and procedures for patient care and safety.\\
-The scope of an e-commerce website might include developing a user-friendly interface, integrating with a payment gateway, establishing a product catalog, and setting up a secure database for customer information.
\subsection{Limitations:} -The limitations of a hospital project might include constraints on the budget, timeline, or available resources, such as the availability of land, equipment, or qualified staff.\\
-The limitations of an e-commerce website might include technical constraints, such as compatibility with different devices and browsers, or legal constraints, such as compliance with data privacy and security regulations.
\section{Business context \label{Section::Business context}}

\quad An In-Hospital Management System (IHMS) is a software application designed to support the operations of a healthcare facility, including patient care, medical records management, resource allocation, and administrative tasks.\\

In today's healthcare landscape, hospitals and clinics are facing increased pressure to provide high-quality patient care while maintaining operational efficiency. The use of IHMS can help address these challenges by streamlining processes and improving communication and collaboration between healthcare providers.\\
Here are some key business contexts for implementing an IHMS:\\
-Improved patient care: With an IHMS, healthcare providers can access up-to-date patient information and medical records in real-time, leading to improved patient outcomes and reduced medical errors.\\

-Increased efficiency: The automation of administrative tasks such as appointment scheduling and medication management can save time and increase operational efficiency.\\

-Better resource allocation: By having a centralized system for managing patient information and resources, healthcare facilities can allocate resources more effectively and reduce waste.\\

-Enhanced communication: With an IHMS, healthcare providers can communicate and collaborate more effectively, leading to improved patient outcomes and reduced medical errors.\\

-Improved compliance: An IHMS can help healthcare facilities stay compliant with industry regulations and standards by tracking and reporting on key metrics such as patient care and resource utilization.\\


Overall, the implementation of an IHMS can help healthcare facilities improve patient care, increase operational efficiency, and stay compliant with industry regulations, making it an attractive solution for hospitals and clinics looking to improve their operations.

\bibliography{bibfile}
%\bibliographystyle{unsrt}
\bibliographystyle{IEEEtran}

%% Initial version by Darian Muresan, Ph.D.
% Edit and adjust as needed.

\documentclass[12pt]{cornell}

% add index support
\makeindex

% graphing programs
\usepackage{color}
\usepackage{psfrag}
\usepackage{verbatim}
\usepackage{fancyhdr}
%\usepackage{titlesec}
\usepackage{fancyvrb} 
% hyperlink programs
\usepackage[pdfmark, 
breaklinks=true, 
colorlinks=true,
citecolor=blue,
linkcolor=blue,
menucolor=black,
pagecolor=black,
urlcolor=blue
]{hyperref} % links in pdf
%\usepackage[colorlinks]{hyperref} % links in dvi
\usepackage{listings}
\usepackage{amsfonts} 
\usepackage{amssymb} 
%\usepackage{tabto}

\usepackage{tabularx,colortbl}
\usepackage[chapter]{algorithm} 
\usepackage{algorithmic} 
\usepackage{blindtext}
\usepackage{imakeidx}


\definecolor{DarkGreen}{rgb}{0,0.6,0}
\definecolor{mygreen}{rgb}{0,0.6,0}
\definecolor{mygray}{rgb}{0.5,0.5,0.5}
\definecolor{mymauve}{rgb}{0.58,0,0.82}

\usepackage{tocloft}
\usepackage{amsmath}
\usepackage{tcolorbox}
\usepackage{enumitem}
\usepackage{longtable}
%\usepackage{textcomp}
\usepackage{txfonts}

%part for \part titles
%chap for \chapter titles
%sec for \section titles
%subsec for \subsection titles
%subsubsec for \subsubsection titles
%para for \paragraph titles
%subpara for \subparagraph titles
%fig for figure \caption titles
%subfig for subfigure \caption titles
%tab for table \caption titles
%subtab for subtable \caption titles

% update chapter number spacing
\setlength{\cftchapnumwidth}{2em}
\setlength{\cftsecnumwidth}{2.5em}
\setlength{\cftsubsecnumwidth}{3.5em}
\setlength{\cftsubsubsecnumwidth}{4.5em}

\addtolength{\cftsecindent}{0.5em}
\addtolength{\cftsubsecindent}{0.5em}
\addtolength{\cftsubsubsecindent}{0.5em}

%\titlespacing*{\chapter}{0pt}{-50pt}{20pt}
%\titleformat{\chapter}[display]{\normalfont\huge\bfseries}{\chaptertitlename\ 
%\thechapter}{20pt}{\Huge}
%\pagestyle{fancy}
%\pagestyle{cornell}
%
%\rhead{F054-021-0172}
%\chead{Nonlinear Enhancement of Visual Target Detection (AF05-T021)}
%\lhead{GSTI}
%\lfoot{\scriptsize Use or disclosure of data on this page is subject
%to the restriction on the title page of this proposal.}
%\cfoot{}
%\rfoot{\thepage}

\newfont{\Bp}{msbm10}
\newfont{\BpBig}{msbm10 scaled\magstep2}
\newfont{\Sc}{eusm10}
\newfont{\ScBig}{eusm10 scaled\magstep3}
\newfont{\Fr}{eufm10}
\newfont{\FrBig}{eufm10 scaled\magstep1}

% some commands:
\newcommand{\dxi}{{\tt m\_xDeltaInput}}
\newcommand{\dyi}{{\tt m\_yDeltaInput}}
\newcommand{\dci}{{\tt m\_cDeltaInput}}
\newcommand{\dxo}{{\tt m\_xDeltaOutput}}
\newcommand{\dyo}{{\tt m\_yDeltaOutput}}
\newcommand{\dco}{{\tt m\_cDeltaOutput}}
\newcommand{\ttf}[1]{{\tt #1}}
\newcommand{\tbl}[2]{{\begin{tabular}{c} #1 \\ #2 \end{tabular}}}

\newcommand{\urltwo}[2]{\mbox{\href{#1}{\tt #2}}}
\newcommand{\qnorm}[1]{\|#1\|_{\bQ}}
\newcommand{\qdot}[2]{\lrb #1, #2 \rrb_{\bQ}}
\newcommand{\kdot}[2]{\lrb #1, #2 \rrb_{\bf k}}
\newcommand{\tdot}[2]{\lrb #1, #2 \rrb}
\newcommand{\mydiff}[2]{\lrb #1 - #2 \rrb}
\newcommand{\lena}{\textit{lena}}
\newcommand{\barb}{\textit{barbara}}
\newcommand{\boat}{\textit{boat}}
\newcommand{\leaves}{\textit{leaves}}
\newcommand{\rings}{\textit{rings}}
\newcommand{\treg}{\textit{train region}}
\newcommand{\dreg}{\textit{denoise region}}
\newcommand{\oreg}{\textit{overlap region}}
\newcommand{\sil}{\sigma_l^2}
\newcommand{\sn}{\sigma^2}
\newcommand{\bn}{{\mbox{\bf \FrBig N}}}
\newcommand{\n}{\mbox{\Fr N}}
%\newcommand{\bn}{\bf N}
%\newcommand{\n}{N}
\newcommand{\bY}{\textbf{Y}}
\newcommand{\bX}{\textbf{X}}
\newcommand{\bb}{\textbf{b}}
\newcommand{\bu}{\textbf{u}}
\newcommand{\bv}{\textbf{v}}
\newcommand{\by}{\textbf{y}}
\newcommand{\bx}{\textbf{x}}
\newcommand{\be}{\textbf{e}}
\newcommand{\bz}{\textbf{z}}
\newcommand{\bs}{\textbf{s}}
\newcommand{\bw}{\textbf{w}}
\newcommand{\bQ}{\textbf{Q}}
\newcommand{\bphi}{\textbf{$\phi$}}
\newcommand{\lsb}{\left[}
\newcommand{\rsb}{\right]}
\newcommand{\lrb}{\left(}
\newcommand{\rrb}{\right)}
\newcommand{\lcb}{\left\{}
\newcommand{\rcb}{\right\}}
\newcommand{\R}{\mbox{\BpBig R}}
\newcommand{\F}{{\cal F}}
\newcommand{\Fk}{\mbox{\Sc F}}
\newcommand{\bQF}{\textbf{Q}_{\mbox{\Sc F}}}
\newcommand{\N}{{\cal N}}
\newcommand{\xlz}{X_l(z)}
\newcommand{\xhz}{X_h(z)}
\newcommand{\xz}{X(z)}
\newcommand{\pr}{ perfect reconstruction }
\newcommand{\smb}{Smith-Barnwell }
\newcommand{\xw}{X(e^{j\omega})}
\newcommand{\xmw}{X(-e^{j\omega})}
\newcommand{\dw}{D(e^{j\omega})}
\newcommand{\dmw}{D(-e^{j\omega})}
\newcommand{\ew}{E(e^{j\omega})}
\newcommand{\emw}{E(-e^{j\omega})}
\newcommand{\fw}{F_0(e^{j\omega})}
\newcommand{\fmw}{F_0(-e^{j\omega})}
\newcommand{\hoz}{H_1(z)}
\newcommand{\hzz}{H_0(z)}
\newcommand{\goz}{G_1(z)}
\newcommand{\gzz}{G_0(z)}
\newcommand{\hzw}{H_{0}(e^{j\omega})}
\newcommand{\hzmw}{H_{0}(-e^{j\omega})}
\newcommand{\hzcw}{H_{0}(e^{-j\omega})}
\newcommand{\how}{H_1(e^{j\omega})}
\newcommand{\homw}{H_1(-e^{j\omega})}
\newcommand{\gzw}{G_0(e^{j\omega})}
\newcommand{\gzmw}{G_0(-e^{j\omega})}
\newcommand{\gow}{G_1(e^{j\omega})}
\newcommand{\gomw}{G_1(-e^{j\omega})}
\newcommand{\wl}{e^{-jwL}}
\newcommand{\aqua}{\textit{AQua with OR }}
\newtheorem{theorem}{Theorem}
\newtheorem{lemma}{Lemma}
\newtheorem{corollary}{Corollary}
\newtheorem{claim}{Claim}
\newtheorem{definition}{Definition}
\newenvironment{proof}{\noindent{\em Proof.}}{\ \hfill Q.E.D.}
%\newtheorem{moduleCount}{L}
\newcommand*{\labelfile}[1]{%
  \label{file:#1}%
}

\lstset{ %
  backgroundcolor=\color{white},   % choose the background color; you must add \usepackage{color} or \usepackage{xcolor}
  basicstyle=\footnotesize,        % the size of the fonts that are used for the code
  breakatwhitespace=false,         % sets if automatic breaks should only happen at whitespace
  breaklines=true,                 % sets automatic line breaking
  captionpos=b,                    % sets the caption-position to bottom
  commentstyle=\color{DarkGreen},    % comment style
  deletekeywords={...},            % if you want to delete keywords from the given language
  escapeinside={\%*}{*)},          % if you want to add LaTeX within your code
  extendedchars=true,              % lets you use non-ASCII characters; for 8-bits encodings only, does not work with UTF-8
  %frame=single,                   % adds a frame around the code
  keepspaces=true,                 % keeps spaces in text, useful for keeping indentation of code (possibly needs columns=flexible)
  keywordstyle=\color{blue},       % keyword style
  language=C++,                    % the language of the code
  morekeywords={*,...},            % if you want to add more keywords to the set
  numbers=left,                    % where to put the line-numbers; possible values are (none, left, right)
  numbersep=5pt,                   % how far the line-numbers are from the code
  numberstyle=\tiny\color{mygray}, % the style that is used for the line-numbers
  rulecolor=\color{black},         % if not set, the frame-color may be changed on line-breaks within not-black text (e.g. comments (green here))
  showspaces=false,                % show spaces everywhere adding particular underscores; it overrides 'showstringspaces'
  showstringspaces=false,          % underline spaces within strings only
  showtabs=false,                  % show tabs within strings adding particular underscores
  stepnumber=1,                    % the step between two line-numbers. If it's 1, each line will be numbered
  stringstyle=\color{mymauve}     % string literal style
  %tabsize=2,                      % sets default tabsize to 2 spaces
  %caption=\lstname                % show the filename of files included with \lstinputlisting; also try caption instead of title
}

% Uncomment draftcopy to get the word DRAFT boldly across the first page
%   By the way, xdvi won't show it but it will come out when you print
%\usepackage[light,all]{draftcopy}		% DRAFT on first page
%\draftcopySetGrey{.97}
%\draftcopyName{Confidential}{150}
%\draftcopFirstPage{1}

% Uncomment drafthead to get the date and DRAFT in the header of pages
% that are normallly numbered on the top, pages 2-n of each chapter for example
% This doesn't work with centered page numbers: \pagestyle{cornellc}
%\usepackage{drafthead}

% Including selective chapters:
% use this to selectively process chapters, etc.  Put a % in front of
% the sections that you don't want done this time.  Includes are
% used instead of \input so that LaTeX will keep track of chapters and
% pages without processing everything.  Don't let any spaces creep in
% around the words or it will not work!


\includeonly{
prologue,
manIntroduction
}

\makeindex

\begin{document}

\pagenumbering{roman}
\singlespacing
% File: prologue.tex
% Thesis prologue:  Title page, acknowledgements, table of contents,
% list of figures, and list of tables.
%
% this file is to be \include'd after the \begin{document}

% Cornell-style title page
\begin{titlepage}
        \title{In-Hospital Management System}
        \author{Team \\ Stevens.edu }
        \conferraldate{}{\today} \maketitle
\end{titlepage}

% Copyright page
%\begin{copyrightpage}
\makecopyright
%\end{copyrightpage}

% Abstract: the abstract body is pulled from the file abstract.tex;
%  the title is pulled from the \title command in the titlepage section
\begin{abstract}
        %\makeabstitle
        \input abstract      % puts the abstract file here
\end{abstract}

% Biographical information pulled from file bio.tex
%\begin{biosketch} \input bio \end{biosketch}

% Dedication (optional):  pulls information from file dedication.tex
%\begin{dedication} 
%\input dedicate 
%\end{dedication}

% Acknowledgements:  pulls information from file acknow
%\begin{acknowledgements} \input acknow \end{acknowledgements}

% Table of contents
\contentspage

% If you have no tables or figures put a % in front of the list page line
% List of tables
\tablelistpage

% List of figures
\figurelistpage



\setcounter{page}{1}        % set page counter
\pagenumbering{arabic}      % set page number style
\pagestyle{fancy}         % top right page numbers
%\pagestyle{cornell}
%\pagestyle{cornellc}       % centered page numbers, disables drafthead

\renewcommand{\chaptermark}[1]{\markboth{#1}{}}
\renewcommand{\sectionmark}[1]{\markright{#1}{}}

\fancyhead{} % clear all fields

\lhead{Chapter \thechapter}
%\lhead{\thechapter}
\chead{\leftmark}
\rhead{\thepage}


\lfoot{Chapter \thechapter}
\cfoot{\copyright Stevens -- \today \mbox{} -- Project Name}
\rfoot{\thepage}

\renewcommand{\headrulewidth}{0.4pt}
\renewcommand{\footrulewidth}{0.4pt}

%\rhead{F054-021-0172}
%\chead{Nonlinear Enhancement of Visual Target Detection (AF05-T021)}
%\lhead{GSTI}
%\lfoot{\scriptsize Use or disclosure of data on this page is subject
%to the restriction on the title page of this proposal.}
%\cfoot{}
%\rfoot{\thepage}


\singlespacing
\chapter{Introduction \\
\small{\textit{-- Team}} 
\index{Chapter!introduction}
\index{introduction}
\label{Chapter::Introduction}}

% Add a section and label it so that we can reference it later
\section{The Team \label{Section::The Team}}

We are a team of three students and below is a biref introduction about us.\\
\textbf{Tanay Shah:}

I am currently pursuing Master's in Computer Science and this is my 2nd Semester. I completed my Bachelors of Computer Engineering in 2020. I worked as a Software Developer for 2 years in an IT firm located in Gujarat, India. I also did freelancing work for a year which included 15 projects. I have good amount of experience in PHP, Java, Python, JavaScript, NodeJS, Laravel Framework, Zend Framework, AWS Management.

\textbf{Chinamy Dali:}

I am currently pursuing a Master's in Computer Science at Stevens Institute of Technology. I have an experience of just over an year in .NET, java spring and nodejs as a full-stack developer; building web-apps and maintaining large scale backends. I am interested in programming languages and compilers, which have been my focus for my graduate studies. I have been built an interpreter for my own language using OCaml and currently pursing Compiler Design course build a X86 architecture targeted compiler for the same language. 

\textbf{Moksha Dave}

I am pursuing my Master's degree with Computer Science as my major. It's my second semester at Stevens and I've been enjoying every part of it. Coming to my bachelor's, I completed my BE in Information Technology from Gujarat Technological University (GTU) during which I studied courses like Data structures, algorithms, computer networks, operating systems, AI, Computer Vision, Data Visualization and a few more. Overall, I am a fast and an enthusiastic learner and a reliable one as well! Something that people wouldn't know about me yet is, I will give in my 110\% once I decide on something.

 \section{Stakeholders for in-Hospital Management System}
 \subsection{Patients and their families:}They are the primary users of the system and have a direct interest in its functionality and reliability.
 \subsection{Hospital administrators:} They are responsible for managing the hospital's resources, including the management system, and are interested in its efficiency, effectiveness, and cost-effectiveness.
 \subsection{Clinical staff:} Doctors, nurses, and other clinical staff members use the system to manage patient information, schedules, and treatments, and are concerned with its accuracy and ease of use.
 \subsection{IT staff:} They are responsible for the installation, maintenance, and support of the system, and are interested in its technical feasibility and scalability.
 \subsection{Insurance providers:} They have an interest in the information recorded in the system, as it is used to determine patient coverage and payment for services.
 \subsection{Regulatory agencies:} They monitor the quality of healthcare services and have an interest in the accuracy and completeness of patient information recorded in the system.
 \subsection{Finance and accounting departments:} They are concerned with the financial aspects of the hospital, including budgeting, billing, and revenue management. The In-Hospital Management System plays a critical role in these processes and they have an interest in its financial reporting capabilities.

 \section{Possible conflicts between stakeholders}
\subsection{Patients and hospital administrators:} Patients may prioritize convenience and accessibility of healthcare services, while administrators may prioritize cost-effectiveness and resource utilization. For example, patients may prefer to have more testing or treatment options available, while administrators may seek to limit the number of tests or treatments to control costs.
\subsection{Clinical staff and IT staff:} Clinical staff may prioritize the ease of use and reliability of the system, while IT staff may prioritize security and scalability. For example, clinical staff may request more user-friendly interface, while IT staff may focus on implementing stronger security measures.
\subsection{Insurance providers and hospital administrators:} Insurance providers may prioritize cost control and cost sharing, while administrators may prioritize providing the best possible healthcare services to patients. For example, insurance providers may seek to limit the number of tests or treatments covered by insurance, while administrators may seek to provide more comprehensive care to patients.
\subsection{Regulatory agencies and hospital administrators:} Regulatory agencies may prioritize patient safety and quality of care, while administrators may prioritize cost-effectiveness and resource utilization. For example, regulatory agencies may require more strict reporting and documentation standards, while administrators may seek to streamline processes to reduce costs.
\subsection{Clinical staff and supply chain management:} Clinical staff may prioritize the availability of necessary medical supplies and equipment, while supply chain management may prioritize cost control and resource utilization. For example, clinical staff may request additional equipment or supplies to improve patient care, while supply chain management may seek to limit the number of items purchased to control costs.


% add a new page
\newpage
Hi there world!  Here is an example of a note\footnote{Here is a reference 
to Figure \ref{Figure::manAgile} and an indexed keyword\index{keyword}.}
\chapter{Vision and Scope Document for In-Hospital Management  \\
\small{\textit{-- Team}} 
\index{Chapter!introduction}
\index{introduction}
\label{Chapter::Introduction}}

% Add a section and label it so that we can reference it later
\section{Business Requirements\label{Section::Business Requirements}}
\subsection{Patient Management} The system should be able to manage patient registration, appointment scheduling, medical history, lab results, prescriptions, and billing.
\subsection{Staff Management: }The system should be able to manage staff scheduling, payroll, performance evaluation, and training.
\subsection{Inventory Management: } The system should be able to manage the hospital's inventory of medical supplies, equipment, and drugs.
\subsection{Medical Records Management:}  The system should be able to manage patient medical records, including diagnosis, treatment, and test results.
\subsection{Communication and Collaboration} The system should facilitate communication and collaboration between staff members, such as messaging, shared calendars, and task assignment.
\subsection{Integration and Interoperability:} The system should be able to integrate with other hospital systems, such as electronic health records, medical imaging systems, and billing systems, and be inter-operable with other healthcare providers
\subsection{Data Security and Privacy} The system should ensure the security and privacy of patient and hospital data, including compliance with applicable regulations and standards.
\subsection{Reporting and Analytics:} The system should provide real-time reporting and analytics, such as occupancy rates, patient outcomes, and financial performance.
\pagebreak

\section{Scope and Limitations \label{Section::Scope and Limitations}}
Scope refers to the boundaries of what a project or system will include, while limitations refer to the factors that could restrict or hinder the project or system's development or use. Here are some examples of scope and limitations:
\subsection{Scope} -The scope of a project to build a new hospital might include designing and constructing the building, installing medical equipment and systems, hiring and training staff, and establishing policies and procedures for patient care and safety.\\
-The scope of an e-commerce website might include developing a user-friendly interface, integrating with a payment gateway, establishing a product catalog, and setting up a secure database for customer information.
\subsection{Limitations:} -The limitations of a hospital project might include constraints on the budget, timeline, or available resources, such as the availability of land, equipment, or qualified staff.\\
-The limitations of an e-commerce website might include technical constraints, such as compatibility with different devices and browsers, or legal constraints, such as compliance with data privacy and security regulations.
\section{Business context \label{Section::Business context}}

\quad An In-Hospital Management System (IHMS) is a software application designed to support the operations of a healthcare facility, including patient care, medical records management, resource allocation, and administrative tasks.\\

In today's healthcare landscape, hospitals and clinics are facing increased pressure to provide high-quality patient care while maintaining operational efficiency. The use of IHMS can help address these challenges by streamlining processes and improving communication and collaboration between healthcare providers.\\
Here are some key business contexts for implementing an IHMS:\\
-Improved patient care: With an IHMS, healthcare providers can access up-to-date patient information and medical records in real-time, leading to improved patient outcomes and reduced medical errors.\\

-Increased efficiency: The automation of administrative tasks such as appointment scheduling and medication management can save time and increase operational efficiency.\\

-Better resource allocation: By having a centralized system for managing patient information and resources, healthcare facilities can allocate resources more effectively and reduce waste.\\

-Enhanced communication: With an IHMS, healthcare providers can communicate and collaborate more effectively, leading to improved patient outcomes and reduced medical errors.\\

-Improved compliance: An IHMS can help healthcare facilities stay compliant with industry regulations and standards by tracking and reporting on key metrics such as patient care and resource utilization.\\


Overall, the implementation of an IHMS can help healthcare facilities improve patient care, increase operational efficiency, and stay compliant with industry regulations, making it an attractive solution for hospitals and clinics looking to improve their operations.

\bibliography{bibfile}
%\bibliographystyle{unsrt}
\bibliographystyle{IEEEtran}

%% Initial version by Darian Muresan, Ph.D.
% Edit and adjust as needed.

\documentclass[12pt]{cornell}

% add index support
\makeindex

% graphing programs
\usepackage{color}
\usepackage{psfrag}
\usepackage{verbatim}
\usepackage{fancyhdr}
%\usepackage{titlesec}
\usepackage{fancyvrb} 
% hyperlink programs
\usepackage[pdfmark, 
breaklinks=true, 
colorlinks=true,
citecolor=blue,
linkcolor=blue,
menucolor=black,
pagecolor=black,
urlcolor=blue
]{hyperref} % links in pdf
%\usepackage[colorlinks]{hyperref} % links in dvi
\usepackage{listings}
\usepackage{amsfonts} 
\usepackage{amssymb} 
%\usepackage{tabto}

\usepackage{tabularx,colortbl}
\usepackage[chapter]{algorithm} 
\usepackage{algorithmic} 
\usepackage{blindtext}
\usepackage{imakeidx}


\definecolor{DarkGreen}{rgb}{0,0.6,0}
\definecolor{mygreen}{rgb}{0,0.6,0}
\definecolor{mygray}{rgb}{0.5,0.5,0.5}
\definecolor{mymauve}{rgb}{0.58,0,0.82}

\usepackage{tocloft}
\usepackage{amsmath}
\usepackage{tcolorbox}
\usepackage{enumitem}
\usepackage{longtable}
%\usepackage{textcomp}
\usepackage{txfonts}

%part for \part titles
%chap for \chapter titles
%sec for \section titles
%subsec for \subsection titles
%subsubsec for \subsubsection titles
%para for \paragraph titles
%subpara for \subparagraph titles
%fig for figure \caption titles
%subfig for subfigure \caption titles
%tab for table \caption titles
%subtab for subtable \caption titles

% update chapter number spacing
\setlength{\cftchapnumwidth}{2em}
\setlength{\cftsecnumwidth}{2.5em}
\setlength{\cftsubsecnumwidth}{3.5em}
\setlength{\cftsubsubsecnumwidth}{4.5em}

\addtolength{\cftsecindent}{0.5em}
\addtolength{\cftsubsecindent}{0.5em}
\addtolength{\cftsubsubsecindent}{0.5em}

%\titlespacing*{\chapter}{0pt}{-50pt}{20pt}
%\titleformat{\chapter}[display]{\normalfont\huge\bfseries}{\chaptertitlename\ 
%\thechapter}{20pt}{\Huge}
%\pagestyle{fancy}
%\pagestyle{cornell}
%
%\rhead{F054-021-0172}
%\chead{Nonlinear Enhancement of Visual Target Detection (AF05-T021)}
%\lhead{GSTI}
%\lfoot{\scriptsize Use or disclosure of data on this page is subject
%to the restriction on the title page of this proposal.}
%\cfoot{}
%\rfoot{\thepage}

\newfont{\Bp}{msbm10}
\newfont{\BpBig}{msbm10 scaled\magstep2}
\newfont{\Sc}{eusm10}
\newfont{\ScBig}{eusm10 scaled\magstep3}
\newfont{\Fr}{eufm10}
\newfont{\FrBig}{eufm10 scaled\magstep1}

% some commands:
\newcommand{\dxi}{{\tt m\_xDeltaInput}}
\newcommand{\dyi}{{\tt m\_yDeltaInput}}
\newcommand{\dci}{{\tt m\_cDeltaInput}}
\newcommand{\dxo}{{\tt m\_xDeltaOutput}}
\newcommand{\dyo}{{\tt m\_yDeltaOutput}}
\newcommand{\dco}{{\tt m\_cDeltaOutput}}
\newcommand{\ttf}[1]{{\tt #1}}
\newcommand{\tbl}[2]{{\begin{tabular}{c} #1 \\ #2 \end{tabular}}}

\newcommand{\urltwo}[2]{\mbox{\href{#1}{\tt #2}}}
\newcommand{\qnorm}[1]{\|#1\|_{\bQ}}
\newcommand{\qdot}[2]{\lrb #1, #2 \rrb_{\bQ}}
\newcommand{\kdot}[2]{\lrb #1, #2 \rrb_{\bf k}}
\newcommand{\tdot}[2]{\lrb #1, #2 \rrb}
\newcommand{\mydiff}[2]{\lrb #1 - #2 \rrb}
\newcommand{\lena}{\textit{lena}}
\newcommand{\barb}{\textit{barbara}}
\newcommand{\boat}{\textit{boat}}
\newcommand{\leaves}{\textit{leaves}}
\newcommand{\rings}{\textit{rings}}
\newcommand{\treg}{\textit{train region}}
\newcommand{\dreg}{\textit{denoise region}}
\newcommand{\oreg}{\textit{overlap region}}
\newcommand{\sil}{\sigma_l^2}
\newcommand{\sn}{\sigma^2}
\newcommand{\bn}{{\mbox{\bf \FrBig N}}}
\newcommand{\n}{\mbox{\Fr N}}
%\newcommand{\bn}{\bf N}
%\newcommand{\n}{N}
\newcommand{\bY}{\textbf{Y}}
\newcommand{\bX}{\textbf{X}}
\newcommand{\bb}{\textbf{b}}
\newcommand{\bu}{\textbf{u}}
\newcommand{\bv}{\textbf{v}}
\newcommand{\by}{\textbf{y}}
\newcommand{\bx}{\textbf{x}}
\newcommand{\be}{\textbf{e}}
\newcommand{\bz}{\textbf{z}}
\newcommand{\bs}{\textbf{s}}
\newcommand{\bw}{\textbf{w}}
\newcommand{\bQ}{\textbf{Q}}
\newcommand{\bphi}{\textbf{$\phi$}}
\newcommand{\lsb}{\left[}
\newcommand{\rsb}{\right]}
\newcommand{\lrb}{\left(}
\newcommand{\rrb}{\right)}
\newcommand{\lcb}{\left\{}
\newcommand{\rcb}{\right\}}
\newcommand{\R}{\mbox{\BpBig R}}
\newcommand{\F}{{\cal F}}
\newcommand{\Fk}{\mbox{\Sc F}}
\newcommand{\bQF}{\textbf{Q}_{\mbox{\Sc F}}}
\newcommand{\N}{{\cal N}}
\newcommand{\xlz}{X_l(z)}
\newcommand{\xhz}{X_h(z)}
\newcommand{\xz}{X(z)}
\newcommand{\pr}{ perfect reconstruction }
\newcommand{\smb}{Smith-Barnwell }
\newcommand{\xw}{X(e^{j\omega})}
\newcommand{\xmw}{X(-e^{j\omega})}
\newcommand{\dw}{D(e^{j\omega})}
\newcommand{\dmw}{D(-e^{j\omega})}
\newcommand{\ew}{E(e^{j\omega})}
\newcommand{\emw}{E(-e^{j\omega})}
\newcommand{\fw}{F_0(e^{j\omega})}
\newcommand{\fmw}{F_0(-e^{j\omega})}
\newcommand{\hoz}{H_1(z)}
\newcommand{\hzz}{H_0(z)}
\newcommand{\goz}{G_1(z)}
\newcommand{\gzz}{G_0(z)}
\newcommand{\hzw}{H_{0}(e^{j\omega})}
\newcommand{\hzmw}{H_{0}(-e^{j\omega})}
\newcommand{\hzcw}{H_{0}(e^{-j\omega})}
\newcommand{\how}{H_1(e^{j\omega})}
\newcommand{\homw}{H_1(-e^{j\omega})}
\newcommand{\gzw}{G_0(e^{j\omega})}
\newcommand{\gzmw}{G_0(-e^{j\omega})}
\newcommand{\gow}{G_1(e^{j\omega})}
\newcommand{\gomw}{G_1(-e^{j\omega})}
\newcommand{\wl}{e^{-jwL}}
\newcommand{\aqua}{\textit{AQua with OR }}
\newtheorem{theorem}{Theorem}
\newtheorem{lemma}{Lemma}
\newtheorem{corollary}{Corollary}
\newtheorem{claim}{Claim}
\newtheorem{definition}{Definition}
\newenvironment{proof}{\noindent{\em Proof.}}{\ \hfill Q.E.D.}
%\newtheorem{moduleCount}{L}
\newcommand*{\labelfile}[1]{%
  \label{file:#1}%
}

\lstset{ %
  backgroundcolor=\color{white},   % choose the background color; you must add \usepackage{color} or \usepackage{xcolor}
  basicstyle=\footnotesize,        % the size of the fonts that are used for the code
  breakatwhitespace=false,         % sets if automatic breaks should only happen at whitespace
  breaklines=true,                 % sets automatic line breaking
  captionpos=b,                    % sets the caption-position to bottom
  commentstyle=\color{DarkGreen},    % comment style
  deletekeywords={...},            % if you want to delete keywords from the given language
  escapeinside={\%*}{*)},          % if you want to add LaTeX within your code
  extendedchars=true,              % lets you use non-ASCII characters; for 8-bits encodings only, does not work with UTF-8
  %frame=single,                   % adds a frame around the code
  keepspaces=true,                 % keeps spaces in text, useful for keeping indentation of code (possibly needs columns=flexible)
  keywordstyle=\color{blue},       % keyword style
  language=C++,                    % the language of the code
  morekeywords={*,...},            % if you want to add more keywords to the set
  numbers=left,                    % where to put the line-numbers; possible values are (none, left, right)
  numbersep=5pt,                   % how far the line-numbers are from the code
  numberstyle=\tiny\color{mygray}, % the style that is used for the line-numbers
  rulecolor=\color{black},         % if not set, the frame-color may be changed on line-breaks within not-black text (e.g. comments (green here))
  showspaces=false,                % show spaces everywhere adding particular underscores; it overrides 'showstringspaces'
  showstringspaces=false,          % underline spaces within strings only
  showtabs=false,                  % show tabs within strings adding particular underscores
  stepnumber=1,                    % the step between two line-numbers. If it's 1, each line will be numbered
  stringstyle=\color{mymauve}     % string literal style
  %tabsize=2,                      % sets default tabsize to 2 spaces
  %caption=\lstname                % show the filename of files included with \lstinputlisting; also try caption instead of title
}

% Uncomment draftcopy to get the word DRAFT boldly across the first page
%   By the way, xdvi won't show it but it will come out when you print
%\usepackage[light,all]{draftcopy}		% DRAFT on first page
%\draftcopySetGrey{.97}
%\draftcopyName{Confidential}{150}
%\draftcopFirstPage{1}

% Uncomment drafthead to get the date and DRAFT in the header of pages
% that are normallly numbered on the top, pages 2-n of each chapter for example
% This doesn't work with centered page numbers: \pagestyle{cornellc}
%\usepackage{drafthead}

% Including selective chapters:
% use this to selectively process chapters, etc.  Put a % in front of
% the sections that you don't want done this time.  Includes are
% used instead of \input so that LaTeX will keep track of chapters and
% pages without processing everything.  Don't let any spaces creep in
% around the words or it will not work!


\includeonly{
prologue,
manIntroduction
}

\makeindex

\begin{document}

\pagenumbering{roman}
\singlespacing
% File: prologue.tex
% Thesis prologue:  Title page, acknowledgements, table of contents,
% list of figures, and list of tables.
%
% this file is to be \include'd after the \begin{document}

% Cornell-style title page
\begin{titlepage}
        \title{In-Hospital Management System}
        \author{Team \\ Stevens.edu }
        \conferraldate{}{\today} \maketitle
\end{titlepage}

% Copyright page
%\begin{copyrightpage}
\makecopyright
%\end{copyrightpage}

% Abstract: the abstract body is pulled from the file abstract.tex;
%  the title is pulled from the \title command in the titlepage section
\begin{abstract}
        %\makeabstitle
        \input abstract      % puts the abstract file here
\end{abstract}

% Biographical information pulled from file bio.tex
%\begin{biosketch} \input bio \end{biosketch}

% Dedication (optional):  pulls information from file dedication.tex
%\begin{dedication} 
%\input dedicate 
%\end{dedication}

% Acknowledgements:  pulls information from file acknow
%\begin{acknowledgements} \input acknow \end{acknowledgements}

% Table of contents
\contentspage

% If you have no tables or figures put a % in front of the list page line
% List of tables
\tablelistpage

% List of figures
\figurelistpage



\setcounter{page}{1}        % set page counter
\pagenumbering{arabic}      % set page number style
\pagestyle{fancy}         % top right page numbers
%\pagestyle{cornell}
%\pagestyle{cornellc}       % centered page numbers, disables drafthead

\renewcommand{\chaptermark}[1]{\markboth{#1}{}}
\renewcommand{\sectionmark}[1]{\markright{#1}{}}

\fancyhead{} % clear all fields

\lhead{Chapter \thechapter}
%\lhead{\thechapter}
\chead{\leftmark}
\rhead{\thepage}


\lfoot{Chapter \thechapter}
\cfoot{\copyright Stevens -- \today \mbox{} -- Project Name}
\rfoot{\thepage}

\renewcommand{\headrulewidth}{0.4pt}
\renewcommand{\footrulewidth}{0.4pt}

%\rhead{F054-021-0172}
%\chead{Nonlinear Enhancement of Visual Target Detection (AF05-T021)}
%\lhead{GSTI}
%\lfoot{\scriptsize Use or disclosure of data on this page is subject
%to the restriction on the title page of this proposal.}
%\cfoot{}
%\rfoot{\thepage}


\singlespacing
\chapter{Introduction \\
\small{\textit{-- Team}} 
\index{Chapter!introduction}
\index{introduction}
\label{Chapter::Introduction}}

% Add a section and label it so that we can reference it later
\section{The Team \label{Section::The Team}}

We are a team of three students and below is a biref introduction about us.\\
\textbf{Tanay Shah:}

I am currently pursuing Master's in Computer Science and this is my 2nd Semester. I completed my Bachelors of Computer Engineering in 2020. I worked as a Software Developer for 2 years in an IT firm located in Gujarat, India. I also did freelancing work for a year which included 15 projects. I have good amount of experience in PHP, Java, Python, JavaScript, NodeJS, Laravel Framework, Zend Framework, AWS Management.

\textbf{Chinamy Dali:}

I am currently pursuing a Master's in Computer Science at Stevens Institute of Technology. I have an experience of just over an year in .NET, java spring and nodejs as a full-stack developer; building web-apps and maintaining large scale backends. I am interested in programming languages and compilers, which have been my focus for my graduate studies. I have been built an interpreter for my own language using OCaml and currently pursing Compiler Design course build a X86 architecture targeted compiler for the same language. 

\textbf{Moksha Dave}

I am pursuing my Master's degree with Computer Science as my major. It's my second semester at Stevens and I've been enjoying every part of it. Coming to my bachelor's, I completed my BE in Information Technology from Gujarat Technological University (GTU) during which I studied courses like Data structures, algorithms, computer networks, operating systems, AI, Computer Vision, Data Visualization and a few more. Overall, I am a fast and an enthusiastic learner and a reliable one as well! Something that people wouldn't know about me yet is, I will give in my 110\% once I decide on something.

 \section{Stakeholders for in-Hospital Management System}
 \subsection{Patients and their families:}They are the primary users of the system and have a direct interest in its functionality and reliability.
 \subsection{Hospital administrators:} They are responsible for managing the hospital's resources, including the management system, and are interested in its efficiency, effectiveness, and cost-effectiveness.
 \subsection{Clinical staff:} Doctors, nurses, and other clinical staff members use the system to manage patient information, schedules, and treatments, and are concerned with its accuracy and ease of use.
 \subsection{IT staff:} They are responsible for the installation, maintenance, and support of the system, and are interested in its technical feasibility and scalability.
 \subsection{Insurance providers:} They have an interest in the information recorded in the system, as it is used to determine patient coverage and payment for services.
 \subsection{Regulatory agencies:} They monitor the quality of healthcare services and have an interest in the accuracy and completeness of patient information recorded in the system.
 \subsection{Finance and accounting departments:} They are concerned with the financial aspects of the hospital, including budgeting, billing, and revenue management. The In-Hospital Management System plays a critical role in these processes and they have an interest in its financial reporting capabilities.

 \section{Possible conflicts between stakeholders}
\subsection{Patients and hospital administrators:} Patients may prioritize convenience and accessibility of healthcare services, while administrators may prioritize cost-effectiveness and resource utilization. For example, patients may prefer to have more testing or treatment options available, while administrators may seek to limit the number of tests or treatments to control costs.
\subsection{Clinical staff and IT staff:} Clinical staff may prioritize the ease of use and reliability of the system, while IT staff may prioritize security and scalability. For example, clinical staff may request more user-friendly interface, while IT staff may focus on implementing stronger security measures.
\subsection{Insurance providers and hospital administrators:} Insurance providers may prioritize cost control and cost sharing, while administrators may prioritize providing the best possible healthcare services to patients. For example, insurance providers may seek to limit the number of tests or treatments covered by insurance, while administrators may seek to provide more comprehensive care to patients.
\subsection{Regulatory agencies and hospital administrators:} Regulatory agencies may prioritize patient safety and quality of care, while administrators may prioritize cost-effectiveness and resource utilization. For example, regulatory agencies may require more strict reporting and documentation standards, while administrators may seek to streamline processes to reduce costs.
\subsection{Clinical staff and supply chain management:} Clinical staff may prioritize the availability of necessary medical supplies and equipment, while supply chain management may prioritize cost control and resource utilization. For example, clinical staff may request additional equipment or supplies to improve patient care, while supply chain management may seek to limit the number of items purchased to control costs.


% add a new page
\newpage
Hi there world!  Here is an example of a note\footnote{Here is a reference 
to Figure \ref{Figure::manAgile} and an indexed keyword\index{keyword}.}
\input{manHomeworkTwo.tex}

\bibliography{bibfile}
%\bibliographystyle{unsrt}
\bibliographystyle{IEEEtran}

%% Initial version by Darian Muresan, Ph.D.
% Edit and adjust as needed.

\documentclass[12pt]{cornell}

% add index support
\makeindex

% graphing programs
\usepackage{color}
\usepackage{psfrag}
\usepackage{verbatim}
\usepackage{fancyhdr}
%\usepackage{titlesec}
\usepackage{fancyvrb} 
% hyperlink programs
\usepackage[pdfmark, 
breaklinks=true, 
colorlinks=true,
citecolor=blue,
linkcolor=blue,
menucolor=black,
pagecolor=black,
urlcolor=blue
]{hyperref} % links in pdf
%\usepackage[colorlinks]{hyperref} % links in dvi
\usepackage{listings}
\usepackage{amsfonts} 
\usepackage{amssymb} 
%\usepackage{tabto}

\usepackage{tabularx,colortbl}
\usepackage[chapter]{algorithm} 
\usepackage{algorithmic} 
\usepackage{blindtext}
\usepackage{imakeidx}


\definecolor{DarkGreen}{rgb}{0,0.6,0}
\definecolor{mygreen}{rgb}{0,0.6,0}
\definecolor{mygray}{rgb}{0.5,0.5,0.5}
\definecolor{mymauve}{rgb}{0.58,0,0.82}

\usepackage{tocloft}
\usepackage{amsmath}
\usepackage{tcolorbox}
\usepackage{enumitem}
\usepackage{longtable}
%\usepackage{textcomp}
\usepackage{txfonts}

%part for \part titles
%chap for \chapter titles
%sec for \section titles
%subsec for \subsection titles
%subsubsec for \subsubsection titles
%para for \paragraph titles
%subpara for \subparagraph titles
%fig for figure \caption titles
%subfig for subfigure \caption titles
%tab for table \caption titles
%subtab for subtable \caption titles

% update chapter number spacing
\setlength{\cftchapnumwidth}{2em}
\setlength{\cftsecnumwidth}{2.5em}
\setlength{\cftsubsecnumwidth}{3.5em}
\setlength{\cftsubsubsecnumwidth}{4.5em}

\addtolength{\cftsecindent}{0.5em}
\addtolength{\cftsubsecindent}{0.5em}
\addtolength{\cftsubsubsecindent}{0.5em}

%\titlespacing*{\chapter}{0pt}{-50pt}{20pt}
%\titleformat{\chapter}[display]{\normalfont\huge\bfseries}{\chaptertitlename\ 
%\thechapter}{20pt}{\Huge}
%\pagestyle{fancy}
%\pagestyle{cornell}
%
%\rhead{F054-021-0172}
%\chead{Nonlinear Enhancement of Visual Target Detection (AF05-T021)}
%\lhead{GSTI}
%\lfoot{\scriptsize Use or disclosure of data on this page is subject
%to the restriction on the title page of this proposal.}
%\cfoot{}
%\rfoot{\thepage}

\newfont{\Bp}{msbm10}
\newfont{\BpBig}{msbm10 scaled\magstep2}
\newfont{\Sc}{eusm10}
\newfont{\ScBig}{eusm10 scaled\magstep3}
\newfont{\Fr}{eufm10}
\newfont{\FrBig}{eufm10 scaled\magstep1}

% some commands:
\newcommand{\dxi}{{\tt m\_xDeltaInput}}
\newcommand{\dyi}{{\tt m\_yDeltaInput}}
\newcommand{\dci}{{\tt m\_cDeltaInput}}
\newcommand{\dxo}{{\tt m\_xDeltaOutput}}
\newcommand{\dyo}{{\tt m\_yDeltaOutput}}
\newcommand{\dco}{{\tt m\_cDeltaOutput}}
\newcommand{\ttf}[1]{{\tt #1}}
\newcommand{\tbl}[2]{{\begin{tabular}{c} #1 \\ #2 \end{tabular}}}

\newcommand{\urltwo}[2]{\mbox{\href{#1}{\tt #2}}}
\newcommand{\qnorm}[1]{\|#1\|_{\bQ}}
\newcommand{\qdot}[2]{\lrb #1, #2 \rrb_{\bQ}}
\newcommand{\kdot}[2]{\lrb #1, #2 \rrb_{\bf k}}
\newcommand{\tdot}[2]{\lrb #1, #2 \rrb}
\newcommand{\mydiff}[2]{\lrb #1 - #2 \rrb}
\newcommand{\lena}{\textit{lena}}
\newcommand{\barb}{\textit{barbara}}
\newcommand{\boat}{\textit{boat}}
\newcommand{\leaves}{\textit{leaves}}
\newcommand{\rings}{\textit{rings}}
\newcommand{\treg}{\textit{train region}}
\newcommand{\dreg}{\textit{denoise region}}
\newcommand{\oreg}{\textit{overlap region}}
\newcommand{\sil}{\sigma_l^2}
\newcommand{\sn}{\sigma^2}
\newcommand{\bn}{{\mbox{\bf \FrBig N}}}
\newcommand{\n}{\mbox{\Fr N}}
%\newcommand{\bn}{\bf N}
%\newcommand{\n}{N}
\newcommand{\bY}{\textbf{Y}}
\newcommand{\bX}{\textbf{X}}
\newcommand{\bb}{\textbf{b}}
\newcommand{\bu}{\textbf{u}}
\newcommand{\bv}{\textbf{v}}
\newcommand{\by}{\textbf{y}}
\newcommand{\bx}{\textbf{x}}
\newcommand{\be}{\textbf{e}}
\newcommand{\bz}{\textbf{z}}
\newcommand{\bs}{\textbf{s}}
\newcommand{\bw}{\textbf{w}}
\newcommand{\bQ}{\textbf{Q}}
\newcommand{\bphi}{\textbf{$\phi$}}
\newcommand{\lsb}{\left[}
\newcommand{\rsb}{\right]}
\newcommand{\lrb}{\left(}
\newcommand{\rrb}{\right)}
\newcommand{\lcb}{\left\{}
\newcommand{\rcb}{\right\}}
\newcommand{\R}{\mbox{\BpBig R}}
\newcommand{\F}{{\cal F}}
\newcommand{\Fk}{\mbox{\Sc F}}
\newcommand{\bQF}{\textbf{Q}_{\mbox{\Sc F}}}
\newcommand{\N}{{\cal N}}
\newcommand{\xlz}{X_l(z)}
\newcommand{\xhz}{X_h(z)}
\newcommand{\xz}{X(z)}
\newcommand{\pr}{ perfect reconstruction }
\newcommand{\smb}{Smith-Barnwell }
\newcommand{\xw}{X(e^{j\omega})}
\newcommand{\xmw}{X(-e^{j\omega})}
\newcommand{\dw}{D(e^{j\omega})}
\newcommand{\dmw}{D(-e^{j\omega})}
\newcommand{\ew}{E(e^{j\omega})}
\newcommand{\emw}{E(-e^{j\omega})}
\newcommand{\fw}{F_0(e^{j\omega})}
\newcommand{\fmw}{F_0(-e^{j\omega})}
\newcommand{\hoz}{H_1(z)}
\newcommand{\hzz}{H_0(z)}
\newcommand{\goz}{G_1(z)}
\newcommand{\gzz}{G_0(z)}
\newcommand{\hzw}{H_{0}(e^{j\omega})}
\newcommand{\hzmw}{H_{0}(-e^{j\omega})}
\newcommand{\hzcw}{H_{0}(e^{-j\omega})}
\newcommand{\how}{H_1(e^{j\omega})}
\newcommand{\homw}{H_1(-e^{j\omega})}
\newcommand{\gzw}{G_0(e^{j\omega})}
\newcommand{\gzmw}{G_0(-e^{j\omega})}
\newcommand{\gow}{G_1(e^{j\omega})}
\newcommand{\gomw}{G_1(-e^{j\omega})}
\newcommand{\wl}{e^{-jwL}}
\newcommand{\aqua}{\textit{AQua with OR }}
\newtheorem{theorem}{Theorem}
\newtheorem{lemma}{Lemma}
\newtheorem{corollary}{Corollary}
\newtheorem{claim}{Claim}
\newtheorem{definition}{Definition}
\newenvironment{proof}{\noindent{\em Proof.}}{\ \hfill Q.E.D.}
%\newtheorem{moduleCount}{L}
\newcommand*{\labelfile}[1]{%
  \label{file:#1}%
}

\lstset{ %
  backgroundcolor=\color{white},   % choose the background color; you must add \usepackage{color} or \usepackage{xcolor}
  basicstyle=\footnotesize,        % the size of the fonts that are used for the code
  breakatwhitespace=false,         % sets if automatic breaks should only happen at whitespace
  breaklines=true,                 % sets automatic line breaking
  captionpos=b,                    % sets the caption-position to bottom
  commentstyle=\color{DarkGreen},    % comment style
  deletekeywords={...},            % if you want to delete keywords from the given language
  escapeinside={\%*}{*)},          % if you want to add LaTeX within your code
  extendedchars=true,              % lets you use non-ASCII characters; for 8-bits encodings only, does not work with UTF-8
  %frame=single,                   % adds a frame around the code
  keepspaces=true,                 % keeps spaces in text, useful for keeping indentation of code (possibly needs columns=flexible)
  keywordstyle=\color{blue},       % keyword style
  language=C++,                    % the language of the code
  morekeywords={*,...},            % if you want to add more keywords to the set
  numbers=left,                    % where to put the line-numbers; possible values are (none, left, right)
  numbersep=5pt,                   % how far the line-numbers are from the code
  numberstyle=\tiny\color{mygray}, % the style that is used for the line-numbers
  rulecolor=\color{black},         % if not set, the frame-color may be changed on line-breaks within not-black text (e.g. comments (green here))
  showspaces=false,                % show spaces everywhere adding particular underscores; it overrides 'showstringspaces'
  showstringspaces=false,          % underline spaces within strings only
  showtabs=false,                  % show tabs within strings adding particular underscores
  stepnumber=1,                    % the step between two line-numbers. If it's 1, each line will be numbered
  stringstyle=\color{mymauve}     % string literal style
  %tabsize=2,                      % sets default tabsize to 2 spaces
  %caption=\lstname                % show the filename of files included with \lstinputlisting; also try caption instead of title
}

% Uncomment draftcopy to get the word DRAFT boldly across the first page
%   By the way, xdvi won't show it but it will come out when you print
%\usepackage[light,all]{draftcopy}		% DRAFT on first page
%\draftcopySetGrey{.97}
%\draftcopyName{Confidential}{150}
%\draftcopFirstPage{1}

% Uncomment drafthead to get the date and DRAFT in the header of pages
% that are normallly numbered on the top, pages 2-n of each chapter for example
% This doesn't work with centered page numbers: \pagestyle{cornellc}
%\usepackage{drafthead}

% Including selective chapters:
% use this to selectively process chapters, etc.  Put a % in front of
% the sections that you don't want done this time.  Includes are
% used instead of \input so that LaTeX will keep track of chapters and
% pages without processing everything.  Don't let any spaces creep in
% around the words or it will not work!


\includeonly{
prologue,
manIntroduction
}

\makeindex

\begin{document}

\pagenumbering{roman}
\singlespacing
\include{prologue}

\setcounter{page}{1}        % set page counter
\pagenumbering{arabic}      % set page number style
\pagestyle{fancy}         % top right page numbers
%\pagestyle{cornell}
%\pagestyle{cornellc}       % centered page numbers, disables drafthead

\renewcommand{\chaptermark}[1]{\markboth{#1}{}}
\renewcommand{\sectionmark}[1]{\markright{#1}{}}

\fancyhead{} % clear all fields

\lhead{Chapter \thechapter}
%\lhead{\thechapter}
\chead{\leftmark}
\rhead{\thepage}


\lfoot{Chapter \thechapter}
\cfoot{\copyright Stevens -- \today \mbox{} -- Project Name}
\rfoot{\thepage}

\renewcommand{\headrulewidth}{0.4pt}
\renewcommand{\footrulewidth}{0.4pt}

%\rhead{F054-021-0172}
%\chead{Nonlinear Enhancement of Visual Target Detection (AF05-T021)}
%\lhead{GSTI}
%\lfoot{\scriptsize Use or disclosure of data on this page is subject
%to the restriction on the title page of this proposal.}
%\cfoot{}
%\rfoot{\thepage}


\singlespacing
\include{manIntroduction}

\bibliography{bibfile}
%\bibliographystyle{unsrt}
\bibliographystyle{IEEEtran}

%\input{manual.ind}
\printindex
\end{document}

\printindex
\end{document}

\printindex
\end{document}

\printindex
\end{document}
